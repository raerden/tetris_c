\documentclass[12pt, letterpaper, twoside]{article}
\usepackage{fontspec}
\usepackage{polyglossia}
\setdefaultlanguage{russian}
\setmainfont{DejaVu Serif}

\title{Документация проекта\\BrickGame Tetris}
\author{WoodsJul. School 21}
\date{October 2025}

\begin{document}

\maketitle
\section{Введение}
Проект BrickGame v1.0 Tetris представляет собой реализацию классической аркадной игры Тетрис с использованием языка программирования C и библиотеки ncurses для терминального интерфейса.

\section{Структура проекта}

\subsection{Библиотека Tetris {src/brick\_game/tetris}}

\begin{itemize}
    \item Файлы с исходным кодом, реализующие логику игры Тетрис.
    \item Функции для взаимодействия с игровым полем, управлением фигурами, проверкой коллизий и обработкой ввода пользователя.
    \item Реализация конечного автомата для формализации логики игры.
\end{itemize}

\subsection{Терминальный интерфейс {src/gui/cli}}

\begin{itemize}
    \item Файлы с исходным кодом, отвечающие за отрисовку игры в терминале с использованием библиотеки ncurses.
    \item Реализация отрисовки игрового поля, чтение ввода пользователя и отображения текущего состояния игры.
\end{itemize}

\section{Сборка проекта}

Сборка осуществляется линковщиком {\textbf{make}} с Makefile, включающим следующие цели:

\begin{itemize}
    \item \textbf{all}: Сборка проекта.
    \item \textbf{install}: Установка программы в систему.
    \item \textbf{uninstall}: Удаление программы из системы.
    \item \textbf{clean}: Очистка временных файлов и папок.
    \item \textbf{dvi}: Создание файла DVI.
    \item \textbf{dist}: Создание архива, содержащего необходимые файлы для сборки и запуска программы.
\end{itemize}

\section{Требования к системе}

Программа разработана на языке С стандарта C11 с использованием компилятора gcc и библиотеки ncurses для терминального интерфейса.

\section{Установка и запуск}

\begin{enumerate}
    \item \textbf{Установка зависимостей:}
        \begin{itemize}
            \item Убедитесь, что у вас установлен компилятор gcc.
            \item Установите библиотеку ncurses.
        \end{itemize}
    \item \textbf{Сборка проекта:}
        \begin{itemize}
            \item Выполните \textbf{make all} для сборки проекта.
        \end{itemize}
    \item \textbf{Установка:}
        \begin{itemize}
            \item Выполните \textbf{make install} для установки программы в систему.
        \end{itemize}
    \item \textbf{Запуск:}
        \begin{itemize}
            \item Выполните \textbf{make run} для запуска программы.
        \end{itemize}
\end{enumerate}

\section{Использование программы}

\begin{enumerate}
    \item \textbf{Управление:}
        \begin{itemize}
            \item Стрелки клавиатуры влево и вправо перемещают фигуру по горизонтали.
            \item Клавишу вниз ускоренно перемещает фигуру вниз.
            \item Для поворота фигуры используйте space.
            \item Для постановки игры на паузу кнопка P.
            \item Для снятия игры с паузы повторно используйте P.
            \item Для выхода из игры используйте ESC.
        \end{itemize}
    \item \textbf{Механики игры:}
        \begin{itemize}
            \item Вращение и перемещение фигур.
            \item Ускорение падения фигуры.
            \item Показ следующей фигуры.
            \item Удаление заполненных линий.
            \item Прогрессивный подсчет очков.
            \item Механика увровней при каждом наборе 600 очков.
            \item Увеличение скорости автоматического движения фигур в зависимости от уровня.
        \end{itemize}
    \item \textbf{Завершение игры:}
        \begin{itemize}
            \item Игра завершается, когда достигнута верхняя граница игрового поля.
            \item Игра завершается победой при достижении 10го уровня.
        \end{itemize}
\end{enumerate}

\section{Тестирование}

Проект включает в себя unit-тесты с использованием библиотеки check. Покрытие библиотеки тестами составляет не менее 80\%.

\end{document}